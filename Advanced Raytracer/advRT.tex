\documentclass[addpoints]{exam}

\usepackage{hyperref}

% Header and footer.
\pagestyle{headandfoot}
\runningheadrule
\runningfootrule
\runningheader{CS 440}{Project II}{Fall 2020}
\runningfooter{}{Page \thepage\ of \numpages}{}
\firstpageheader{}{}{}

\qformat{{\large\bf \thequestion. \thequestiontitle}\hfill}
\boxedpoints
% \printanswers

\title{Project II}
\author{CS 440 Computer Graphics\\Habib University\\Fall 2020}
\date{}

\begin{document}
\maketitle
\thispagestyle{empty}

\begin{flushleft}
  \Large{\underline{\bf II A: Ray Tracing Engine}}\\\smallskip
  \large{Due: 2359h on Tue, 22 Dec, 2020}
\end{flushleft}

In this project you will extend your ray tracing engine from Homework 4.

You may enhance object appearance by implementing new \emph{materials} which have a \emph{BRDF}. You may implement anti-aliasing by adding new \textit{samplers}. The scene will be lit by one or more \emph{light sources}. Ray-tracing is accelerated through \textit{acceleration structures}. Secondary and shadow rays are traced through various \textit{tracers}.

You will correspondingly derive new classes from \texttt{Material} and \texttt{Sampler}, construct new hierarchies for \texttt{BRDF}, \texttt{Light}, \texttt{Acceleration}, and \texttt{Tracer}.

\begin{questions}
  
  \titledquestion{Appearance}
  Implement some \texttt{BRDF}s and correspondingly define new \texttt{Material} subclasses that use them.

  \titledquestion{Lighting}
  The Cosine shader in Homework 4 corresponds to a light source at the camera. Implement other light sources, e.g. point, spotlight, and directional by adding a new folder called \texttt{lights} and populating it with a hierarchy of \texttt{Light} classes. Add a \texttt{std::vector<Light*>} member to \texttt{World} and use it to light the scene.

  \titledquestion{Acceleration}
  Add a new folder called \texttt{acceleration} and populate it with a hierarchy of acceleration structures. Add an \texttt{Acceleration*} member to \texttt{World} and use it to compute ray intersections.

  \titledquestion{Ray Casting}
  The ray tracer in \texttt{raytracer.cpp} is very basic--it shades based on primary rays only. Add a new folder called \texttt{tracers} and populate it with a hierarchy of \texttt{Tracer} classes. You can start the hierarchy by moving the ray tracer in \texttt{raytracer.cpp} to a \texttt{Basic} class derived from \texttt{Tracer}. Add other ray tracers that implement other ray tracing features like shadows, recursive levels of reflection, and transparency. Add a \texttt{Tracer*} member to \texttt{World} and use it for ray tracing.

  \newpage
  \titledquestion{Showcase}
  We now want to showcase your ray tracing engine--how good it is and how it can be used to create stunning images. Your task is to:
  \begin{itemize}
  \item \textbf{Create an original scene}. The scene should be \emph{original}. You may derive inspiration from past rendering competitions such as those listed on the LMS page containing \href{https://hulms.instructure.com/courses/332/pages/raytracing-links?module_item_id=18597}{Raytracing Links}, but the final scene should be the product of your own imagination.
    
    You may use third-party assets. The used assets must be publicly available for free. You can use them to build your original scene but it is not allowed for the whole, or major part, of the scene to be reused from somewhere. Alternatively, you can model everything yourself. Please see the LMS page on \href{https://hulms.instructure.com/courses/332/pages/3d-models-and-modeling?module_item_id=18598}{3D Models and Modeling} for related resources.

  \item \textbf{Use your engine to render the scene}. You must create two renders of your scene: a low quality render at a resolution of 480x360 and a high quality render with resolution 1920x1080 or higher. You may use different aspect ratios if they better fit your scene.
    
    The images must be rendered by your engine. Any post-processing must be implemented within your framework. The images need not be realistic. You can use features that do not follow real world physics if it better suits your artistic concept.
    
    The high quality image must render in less than 6 hours on a modern computer.

  \item \textbf{Create a web page to showcase your work}. The website should feature your render and its title. It should include the following.
    \begin{description}
    \item[Concept] description of your concept and how you arrived at it,
    \item[Scene] describe how you built your scene,
    \item[Image Features] highlight interesting parts or features of your render. Additional images may be included for this purpose.
    \item[Code Features] list all the features you have implemented in your ray tracer. This includes changes made for Homework 5 and for this project.
    \item[Acceleration] include a table comparing rendering times of your ray tracer with and without an acceleration structure. Supporting renderings must be included.
    \item[Build] for every image included on the page, a link to the corresponding implementation of {\tt World::build},
    \item[Acknowledgment] acknowledge all third party sources of used assets or resources, once where they are used, e.g. in the caption of a rendered image, and again in an \textbf{Acknowledgment} section toward the bottom of the page.
    \item[Team] include the names of all team members and a photograph of your team.
    \item[Comments] include any other comments desired by the team.
    \end{description}
  \end{itemize}

  Some sample webpages are provided on LMS under \href{https://hulms.instructure.com/courses/332/files/66380/download?download_frd=1}{Project II Showcase Samples}. Any accompanying build or code files have been removed from the samples. As your submission may also be shared in the future as a sample, take care to only include images which you are comfortable sharing publicly.
  
\end{questions}

\section*{Credits}

This project is adapted from the rendering competition run by \href{https://graphics.cg.uni-saarland.de/people/slusallek.html}{Philipp Slusallek} in his \href{https://graphics.cg.uni-saarland.de/courses/cg1-2020/}{Computer Graphics 1 course}. The code is adapted from that provided by \href{http://www.raytracegroundup.com/}{Kevin Suffern}.

\newpage


\begin{flushleft}
  \Large{\underline{\bf II B: Creative Expression}}\\\smallskip
  \large{Due: 2359h on Tue, 15 Dec, 2020}
\end{flushleft}

Computer Graphics is a creative activity. It brings together mathematics, computer science, and programming in imaginative ways. And its use as a medium for creative and aesthetic expression needs no introduction.

People's usual interaction with Computer Graphics invokes in them a sense of awe, wonder, excitement, amazement, fascination, curiosity, aesthetic, and inspiration. A similar cocktail of sentiments led you to this course and having engaged with the material intellectually, you may now additionally feel appreciation and motivation toward the field.

For this project, you have to channel your sentiments toward this course and Computer Graphics in general through a piece of creative expression. This is a collaborative effort---you will work in teams. And it is highly personal---the emotions that are invoked and our manner of channeling them are unique to each of us.

This expression can take on a myriad forms and it is expected that whatever you produce is befitting of your four month experience in this course and how you feel. Something so personal and creative can hardly be prescribed but here are some examples for the sake of illustration.
\begin{itemize}
\item a play, fictional or otherwise, about the people involved,
\item a report on further topics not covered in this course,
\item you can go meta and express your sentiments on CG using CG, e.g. through renderings,
\item a poem or essay,
\item a game, digital or otherwise,
\item research into the history of IRTC and the motivations and technical details of its entries,
\item an ode to your GPU,
\item cosplay as your favorite polygon,
\item a piece of music, even better if it is programmed, on your GPU!
\end{itemize}

While sober and somber attempts are completely acceptable, do not be afraid to bring out your crazy! If there is anything we have learned about Computer Graphics, it is that the wild, the daring, the imaginative, the bold, the crazy, the fantastic, and the whimsical are completely at home in this domain. Of course, there must be a sound, justifiable, and demonstrable grounding in Computer Graphics.

Whatever form you choose, it must be amenable to presentation in an online format, e.g. through a YouTube video, and to digital submission, i.e. on GitHub. Your eventual expression must also reflect an effort proportional to the size of your team.

ich lass mich \"uberraschen. viel Spa\ss!

\end{document}

%%% Local Variables:
%%% mode: latex
%%% TeX-master: t
%%% End:
