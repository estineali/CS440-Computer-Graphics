\documentclass{article}

\usepackage{fullpage}
\usepackage{hyperref}

\title{Homework IV: Ray Tracing Engine}
\author{CS 440 Computer Graphics\\Habib University, Fall 2020}
\date{Due: 2359h on Friday, 27 November}

\begin{document}
\maketitle
\thispagestyle{empty}

In this assignment, we will implement an extendable ray tracing engine.

It is important to understand the setup. A scene to be rendered consists of \emph{geometry} whose appearance is determined through \emph{material}s. An image is formed when the scene is \emph{sample}d by a \emph{camera} through a \emph{view plane}. The variety of objects of each of the above types is captured through class hierarchies in the accompanying \texttt{raytracer} directory.
\begin{itemize}
\item The \texttt{geometry} folder contains declarations of an abstract class, \texttt{Geometry}, and concrete classes  \texttt{Plane}, \texttt{Sphere}, and \texttt{Triangle} which inherit from \texttt{Geometry}.
\item The \texttt{materials} folder contains declarations of an abstract class, \texttt{Material}, and a concrete class  \texttt{Cosine} that inherits from \texttt{Material}. The cosine material assigns color based on the angle between the ray and the normal at the hit point.
\item The \texttt{samplers} folder contains declarations of an abstract class, \texttt{Sampler}, and a concrete class  \texttt{Simple} that inherits from \texttt{Sampler}. The \texttt{Simple} sampler shoots 1 ray through the center of a pixel.
\item The \texttt{cameras} folder contains declarations of an abstract class, \texttt{Camera}, and concrete classes, \texttt{Parallel} and \texttt{Perspective}, that inherit from \texttt{Camera}. A parallel camera stores the direction of projection and a perspective camera stores the center of projection.
\end{itemize}

The \texttt{utilities} folder includes declarations of utility classes, notably \texttt{Image} and \texttt{ShadeInfo}. \texttt{Image} holds pixel colors and writes an image to file in \href{https://en.wikipedia.org/wiki/Netpbm_format#PPM_example}{PPM format} (ASCII version). \texttt{ShadeInfo} contains all the information required for shading a point. These classes refer to classes from the \texttt{world} folder.

The \texttt{world} folder includes declarations of 2 classes. \texttt{ViewPlane} contains information on the view plane. \texttt{World} contains all the information required to render the scene--the geometry and associated materials, the view plane, the camera and sampler, and the background color. \texttt{World::build} populates the world. You will reimplement this function each time to define a new image to be rendered. The \texttt{build} folder includes some sample implementations of \texttt{World::build}.

The file \texttt{raytracer.cpp} illustrates how everything comes together in order to render a scene.

Go over the provided files to make sure that you understand the overall structure.
  
\section*{Task}
Your task is to implement the necessary classes such that running the  provided \texttt{raytracer.cpp} as-is renders the scene defined in \texttt{World::build}. You may use the 3 sample \texttt{build} functions provided in the \texttt{build} folder. Share the resulting images in the LMS discussion thread for this assignment.

\section*{Credits}

This project is adapted from the rendering competition run by \href{https://graphics.cg.uni-saarland.de/people/slusallek.html}{Philipp Slusallek} in his \href{https://graphics.cg.uni-saarland.de/courses/cg1-2018/}{Computer Graphics 1 course}. The code is adapted from that provided by \href{http://www.raytracegroundup.com/}{Kevin Suffern}.

\end{document}
%%% Local Variables:
%%% mode: latex
%%% TeX-master: t
%%% End:
